\documentclass{report}
\usepackage[utf8]{inputenc}

\usepackage[italian]{babel}
\usepackage[italian]{cleveref}

\title{OOP-FARGOAL}
\author{
    Bulgarelli Marco, marco.bulgarelli@studio.unibo.it \and 
    Ravaioli Alessandro, alessandro.ravaioli@studio.unibo.it \and
    Tassinari Sabrina, sabrina.tassinari@studio.unibo.it \and
    Tramonti Daniele, daniele.tramonti@studio.unibo.it 
}
\date{15 febbraio 2025}

\begin{document}
\maketitle

\tableofcontents

\chapter{Analisi}

\section{Desctizione e requisiti}

Il software mira alla costruzione di un videogioco ispirato a “Sword of Fargoal”. 
%
Quest’ultimo è un “Dungeon Crawler Arcade”, ovvero basato sull’esplorazione di un labirinto a più piani, con lo scopo di riportare in superficie la Spada di Fargoal attraversando innumerevoli pericoli. 
%
La nostra versione cercherà di essere il più fedele possibile al gioco originale.

\subsection*{Requisiti funzionali}
\begin{itemize}
    \item Il giocatore si muoverà all’interno di una grande stanza che corrisponde ad un piano del Dungeon. La generazione di ogni piano (e dei suoi contenuti) dovrà essere casuale.
    \item Il piano è caratterizzato dalla presenza di mostri. Questi possono apparire in luoghi specifici o in punti casuali, aumentando gradualmente con il passare del tempo, rendendo l’ambiente sempre più pericoloso.
    \item All’interno del piano saranno presenti due tipologie di oggetti di cui il giocatore potrà usufruire: dei bauli, che possono contenere degli oggetti magici o delle trappole, oppure delle sacche di monete.
    \item Il sistema di progresso del personaggio è legato all’accumulo di esperienza, che contribuisce all’aumentare del suo livello. Questa può essere ottenuta sia combattendo contro i mostri che offrendo donazioni in oro ai templi.
    \item In ogni piano sarà presente almeno un tempio, dove è possibile donare oro per salire di livello, rigenerare punti ferita e ottenere una probabilità di essere curati completamente. Inoltre, i templi garantiscono temporanea inattaccabilità ed una rigenerazione accelerata.
    \item Una volta recuperata la spada, è necessario riportarla in superficie, ritornando al primo piano e salendo le scale per uscire dal labirinto. La sfida sta nel fatto che, se il giocatore viene attaccato da un mostro, perde la spada, che ritornerà automaticamente nel piano in cui è stata trovata.
\end{itemize}

\chapter{Design}

\section{Architettura}

\section{Design dettagliato}

\chapter{Sviluppo}

\chapter{Commenti finali}


\end{document}